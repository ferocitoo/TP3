\documentclass[a4paper, 12pt,oneside]{article}
%On peut changer "oneside" en "twoside" si on sait que le résultat sera recto-verso.
%Cela influence les marges (pas ici car elles sont identiques à droite et à gauche)

% pour l'inclusion de figures en eps,pdf,jpg,....
\usepackage{graphicx}
\usepackage{subcaption}

\usepackage{amssymb}

\usepackage{float}
\usepackage{caption}
\usepackage{multirow}



%Marges. Désactiver pour utiliser les valeurs LaTeX par défaut
%\usepackage[top=2.5cm, bottom=2cm, left=2cm, right=2cm, showframe]{geometry}
\usepackage[top=2.5cm, bottom=2cm, left=2cm, right=2cm]{geometry}
% quelques symboles mathematiques en plus
\usepackage{amsmath}

% le tout en langue francaise
%\usepackage[francais]{babel}

% on peut ecrire directement les charactères avec l'accent
\usepackage[T1]{fontenc}

% a utiliser sur Linux/Windows
%\usepackage[latin1]{inputenc}

% a utiliser avec UTF8
\usepackage[utf8]{inputenc}
%Très utiles pour les groupes mixtes mac/PC. Un fichier texte enregistré sous codage UTF-8 est lisible dans les deux environnement.
%Plus de problème de caractères accentués et spéciaux qui ne s'affichent pas

% a utiliser sur le Mac
%\usepackage[applemac]{inputenc}

% pour l'inclusion de liens dans le document (pdflatex)
\usepackage[colorlinks,bookmarks=false,linkcolor=black,urlcolor=blue, citecolor=black]{hyperref}

%Pour l'utilisation plus simple des unités et fractions
\usepackage{units}

%Pour utiliser du time new roman... Comenter pour utiliser du ComputerModern
%\usepackage{mathptmx}

%Pour du code non interprété
\usepackage{verbatim}
\usepackage{verbdef}% http://ctan.org/pkg/verbdef

%Pour changer la taille des titres de section et subsection. Ajoutez manuellement les autres styles si besoin.
\makeatletter
\renewcommand{\section}{\@startsection {section}{1}{\z@}%
             {-3.5ex \@plus -1ex \@minus -.2ex}%
             {2.3ex \@plus.2ex}%
             {\normalfont\normalsize\bfseries}}
\makeatother

\makeatletter
\renewcommand{\subsection}{\@startsection {subsection}{1}{\z@}%
             {-3.5ex \@plus -1ex \@minus -.2ex}%
             {2.3ex \@plus.2ex}%
             {\normalfont\normalsize\bfseries}}
\makeatother

%Début du document
\begin{document}


\begin{center}
\large\textbf{\sffamily Deterministic Chaos}\\%
\large\sffamily Group N$^\circ$1: Alexis Escarmelle, Nil Fajas\\%
\large\sffamily \today\qquad Mathieu Padlewski\\%
\end{center}

%			Introduction
\section{Introduction}
Quantum entanglement is one of the most striking and counterintuitive phenomena in quantum mechanics, challenging classical notions of locality and reality. First formally discussed by Schrödinger in 1935, entanglement describes a situation where the quantum state of a composite system cannot be factorized into independent states of its subsystems \cite{Schrodinger}. This phenomenon was at the heart of a famous debate initiated by Einstein, Podolsky, and Rosen (EPR) in their seminal paper, where they questioned whether quantum mechanics could be considered a complete theory \cite{EPR}. EPR formulated a thought experiment suggesting that quantum mechanics allows for nonlocal correlations between distant particles, leading them to argue in favor of hidden variables that would restore determinism to physical theory.

The implications of entanglement remained largely theoretical until the groundbreaking work of John Bell in 1964, who derived what are now known as Bell's inequalities \cite{Bell}. Bell demonstrated that if hidden-variable theories governed quantum mechanics, the correlations predicted by quantum theory should obey certain mathematical constraints. However, quantum mechanics predicts violations of these inequalities, implying that no local hidden-variable theory can fully describe quantum reality.

The definitive experimental confirmation of quantum entanglement came in 1982 with the pioneering work of Alain Aspect and his collaborators \cite{Alain Aspect}. Their experiment involved measuring polarization correlations in entangled photon pairs, implementing fast-switching detection schemes to close the locality loophole. The results unambiguously demonstrated violations of Bell’s inequalities, providing strong evidence against local realism and supporting the inherently nonlocal nature of quantum mechanics.

The experimental realization of entanglement has since opened avenues for revolutionary applications, including quantum cryptography, quantum computing, and teleportation \cite{applications}.

The goal of this report is to show the violation of the CSHS inequality by quantum entanglement. The CSCH inequality is a particular case of the Bell inequality. To demonstrate this equality, multiple wxperiments will be conducted to measure  the polarization correlations of entangled photon pairs under different experimental conditions. The theroy, experimental setup and results will be presented in the following sections.

\section{Theory}
\subsection{Spontaneous parametric down conversion}
In the following experiments, a laser beam is directed into a nonlinear crystal, where it undergoes a process called spontaneous parametric down conversion (SPDC). In this process, a single photon (called pump photon) from the laser beam with frequency $\omega_p$ is converted into two photons (called signal and idler photons) with frequencies $\omega_s$ and $\omega_i$ respectively. 

The two photons generated are entangled in their polarization states. The nonlinear crystal used is a BBO (baryum $\beta$-borate) crystal, which produces type I phase SPDC. In this type of SPDC, the two entangled photons have the same polarization state. In type II phase SPDC, the two photons would have orthogonal polarization states.

The Hamiltonian of the SPDC process is given by:
\begin{equation}
    \hat{H}_{SPDC} = \kappa (\hat{a_p} \hat{a_s}^\dagger \hat{a_i}^\dagger + \text{h.c.})
\end{equation}
where $\hat{a_p}$, $\hat{a_s}$ and $\hat{a_i}$ are the annihilation operators of the pump, signal and idler photons respectively. They act on the Fock space of the corresponding modes, i.e. 
\begin{equation}
    \hat{a_q} |n_q\rangle = \sqrt{n_q} |n_q - 1\rangle
\end{equation}
\begin{equation}
    \hat{a_q}^\dagger |n_q\rangle = \sqrt{n_q + 1} |n_q + 1\rangle
\end{equation}
with $n_q$ the number of photons in the mode $q$.
The letters h.c. in the Hamiltonian stand for the hermitian conjugate. $\kappa$ is a constant that depends on the crystal properties and the pump beam intensity.

\subsection{Energy and momentum conservation of the SPDC process}
The SPDC process conserves energy and momentum. The energy conservation of a single photon is given by:
\begin{equation}
    E = \hbar \omega 
\end{equation}
where $\omega$ is the frequency of the photon. By energy conservation,
\begin{equation}
    E_p = E_s + E_i \implies \omega_p = \omega_s + \omega_i
\end{equation}
Then, the momentum of a single photon is given by:
\begin{equation}
    \vec{p} = \hbar \vec{k}
\end{equation}
The dispersion relation in a birefringent crystal is given by:
\begin{equation}
    \vec{k} = \omega \vec{n}(\omega,\hat{k})/c
\end{equation} 
where $\vec{n}(\omega,\hat{k})$ is the refractive index of the crystal at frequency $\omega$ and $\hat{k}$  the direction of propagation ($\vec{k} = k \hat{k}$). The vector $\vec{n}(\omega,\hat{k})$ decomposes to $\vec{n}(\omega,\hat{k}) = \tilde{n}(\omega) \hat{k}$, with $\tilde{n}(\omega)$ a tensor that depends on the crystal properties. 

The momentum conservation of the SPDC process is given by:
\begin{equation}
    \vec{p}_p = \vec{p}_s + \vec{p}_i \implies \vec{k}_p = \vec{k}_s + \vec{k}_i
\end{equation}
By the dispersion relation, we have:
\begin{equation}
    \omega_p \tilde{n}_p \hat{k_p} = \omega_s \tilde{n}_s \hat{k_s} + \omega_i \tilde{n}_i \hat{k_i}
\end{equation}
To maximize the efficiency of the SPDC process, phase-matching conditions must be satisfied. Indeed, if there is a phase mismatch in the process, there will be destructive interference and the efficiency of the process will be reduced. To satisfy the phase-matching conditions, 


\section{Results}


%			Bibliographie
\begin{thebibliography}{99}

    \bibitem{Schrodinger}
    Schrödinger, E. (1935, October). Discussion of probability relations between separated systems. In Mathematical Proceedings of the Cambridge Philosophical Society (Vol. 31, No. 4, pp. 555-563). Cambridge University Press.
    
    \bibitem{EPR}
    Einstein, A., Podolsky, B., \& Rosen, N. (1935). Can quantum-mechanical description of physical reality be considered complete?. Physical review, 47(10), 777.
    
    \bibitem{Bell}
    Bell, J. S. (1964). On the einstein podolsky rosen paradox. Physics Physique Fizika, 1(3), 195.

    \bibitem{Alain Aspect}
    Aspect, A., Grangier, P., \& Roger, G. (1982). Experimental realization of Einstein-Podolsky-Rosen-Bohm Gedankenexperiment: a new violation of Bell's inequalities. Physical review letters, 49(2), 91.

    \bibitem{applications}
    Horodecki, R., Horodecki, P., Horodecki, M., \& Horodecki, K. (2009). Quantum entanglement. Reviews of modern physics, 81(2), 865-942.
    
    
\end{thebibliography}
\end{document}
